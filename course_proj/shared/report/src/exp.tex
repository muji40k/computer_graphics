\section{Экспериментальная часть}

\subsection{Технические характеристики}

Технические характеристики устройства, на котором выполнялось тестирование.

\begin{itemize}
    \item Операционная система: Arch 6.1.9 \cite{Arch} Linux \cite{Linux} x86\_64.
    \item Память: 16 Гб.
    \item Процессор: AMD Ryzen™ 5 4600H CPU @ 3.0G ГГц \cite{AMD_CPU}.
\end{itemize}

При тестировании ноутбук был включен в сеть электропитания. Во время
тестирования ноутбук был нагружен только встроенными приложениями окружения,
а также системой тестирования.

\subsection{Постановка эксперимента}

\subsubsection{Цель эксперимента}

Цель эксперимента - оценить время работы программы под влиянием паараметров
таких, как размер изображения, количество потоков, разбиение по
горизонтали/вертикали.

Исходя из ожиданий, с увеличение количества потоков время работы алгоритма
должно сокращаться в опбратной зависимости.

Для подтверждения данной гипотезы необходимо провести теоретические замеры
времени выполнения.

\subsubsection{Проведение замеров}

На рисунке \ref{graph_plain} преведено время работы алгоритма от размера
визуализируемого изображения в случае однопоточной работы приложения.

\begin{figure}[!h]
\begin{tikzpicture}
\begin{axis}[axis lines=left,
             width=0.9\textwidth,
             height=0.4\textheight,
             xlabel={Размер изображения},
             ylabel={Время (с)},
             xmin=0, xmax=210,
             legend pos=north west,
             ymajorgrids=true,
             xmajorgrids=true,
             grid style=dashed]
\addplot table [domain=0:210,
                x=n,
                y=t,
                col sep=comma,
                mark=none]
         {data/plain.csv};
\addlegendentry{Результаты измерений}
\addplot[domain=10:200, mark=none]{0.002467712 * x ^ 2};
\addlegendentry{\(0.002467712~n ^ 2\)}
\end{axis}
\end{tikzpicture}
\caption{Замеры времени работы программы в случае линейного увеличения количества пикселей экрана}
\label{graph_plain}
\end{figure}

\clearpage

На рисунках \ref{graph_10} -- \ref{graph_200} представлены соответсвующие
результаты замеров времени.

\begin{figure}[!h]
\begin{tikzpicture}
\begin{axis}[axis lines=left,
             width=0.9\textwidth,
             height=0.4\textheight,
             xlabel={Число потоков},
             ylabel={Время (с)},
             xmin=0, xmax=12,
             legend pos=north east,
             ymajorgrids=true,
             xmajorgrids=true,
             grid style=dashed]
\addplot table [domain=0:12,
                x=n,
                y=t,
                col sep=comma,
                mark=none, color=red]
         {data/parallel_2_10.csv};
\addlegendentry{Разбиение на 2}
\addplot table [domain=0:12,
                x=n,
                y=t,
                col sep=comma,
                mark=none, color=green]
         {data/parallel_4_10.csv};
\addlegendentry{Разбиение на 4}
\end{axis}
\end{tikzpicture}
\caption{Замеры времени работы программы в случае увеличения числа потоков для
         изображения размером 10 пикселей}
\label{graph_10}
\end{figure}

\begin{figure}[!h]
\begin{tikzpicture}
\begin{axis}[axis lines=left,
             width=0.9\textwidth,
             height=0.4\textheight,
             xlabel={Число потоков},
             ylabel={Время (с)},
             xmin=0, xmax=12,
             legend pos=north east,
             ymajorgrids=true,
             xmajorgrids=true,
             grid style=dashed]
\addplot table [domain=0:12,
                x=n,
                y=t,
                col sep=comma,
                mark=none, color=red]
         {data/parallel_2_57.csv};
\addlegendentry{Разбиение на 2}
\addplot table [domain=0:12,
                x=n,
                y=t,
                col sep=comma,
                mark=none, color=green]
         {data/parallel_4_57.csv};
\addlegendentry{Разбиение на 4}
\addplot table [domain=0:12,
                x=n,
                y=t,
                col sep=comma,
                mark=none, color=blue]
         {data/parallel_6_57.svg};
\addlegendentry{Разбиение на 6}
\addplot table [domain=0:12,
                x=n,
                y=t,
                col sep=comma,
                mark=none, color=brown]
         {data/parallel_8_57.svg};
\addlegendentry{Разбиение на 8}
\addplot table [domain=0:12,
                x=n,
                y=t,
                col sep=comma,
                mark=none, color=magenta]
         {data/parallel_10_57.svg};
\addlegendentry{Разбиение на 10}
\end{axis}
\end{tikzpicture}
\caption{Замеры времени работы программы в случае увеличения числа потоков для
         изображения размером 57 пикселей}
\label{graph_57}
\end{figure}

\clearpage

\begin{figure}[!h]
\begin{tikzpicture}
\begin{axis}[axis lines=left,
             width=0.9\textwidth,
             height=0.4\textheight,
             xlabel={Число потоков},
             ylabel={Время (с)},
             xmin=0, xmax=12,
             legend pos=north east,
             ymajorgrids=true,
             xmajorgrids=true,
             grid style=dashed]
\addplot table [domain=0:12,
                x=n,
                y=t,
                col sep=comma,
                mark=none]
         {data/parallel_2_105.csv};
\addlegendentry{Разбиение на 2}
\addplot table [domain=0:12,
                x=n,
                y=t,
                col sep=comma,
                mark=none]
         {data/parallel_4_105.csv};
\addlegendentry{Разбиение на 4}
\addplot table [domain=0:12,
                x=n,
                y=t,
                col sep=comma,
                mark=none]
         {data/parallel_6_105.svg};
\addlegendentry{Разбиение на 6}
\addplot table [domain=0:12,
                x=n,
                y=t,
                col sep=comma,
                mark=none]
         {data/parallel_8_105.svg};
\addlegendentry{Разбиение на 8}
\addplot table [domain=0:12,
                x=n,
                y=t,
                col sep=comma,
                mark=none]
         {data/parallel_10_105.svg};
\addlegendentry{Разбиение на 10}
\end{axis}
\end{tikzpicture}
\caption{Замеры времени работы программы в случае увеличения числа потоков для
         изображения размером 105 пикселей}
\label{graph_105}
\end{figure}

\begin{figure}[!h]
\begin{tikzpicture}
\begin{axis}[axis lines=left,
             width=0.9\textwidth,
             height=0.4\textheight,
             xlabel={Число потоков},
             ylabel={Время (с)},
             xmin=0, xmax=12,
             legend pos=north east,
             ymajorgrids=true,
             xmajorgrids=true,
             grid style=dashed]
\addplot table [domain=0:12,
                x=n,
                y=t,
                col sep=comma,
                mark=none]
         {data/parallel_2_153.csv};
\addlegendentry{Разбиение на 2}
\addplot table [domain=0:12,
                x=n,
                y=t,
                col sep=comma,
                mark=none]
         {data/parallel_4_153.csv};
\addlegendentry{Разбиение на 4}
\addplot table [domain=0:12,
                x=n,
                y=t,
                col sep=comma,
                mark=none]
         {data/parallel_6_153.svg};
\addlegendentry{Разбиение на 6}
\addplot table [domain=0:12,
                x=n,
                y=t,
                col sep=comma,
                mark=none]
         {data/parallel_8_153.svg};
\addlegendentry{Разбиение на 8}
\addplot table [domain=0:12,
                x=n,
                y=t,
                col sep=comma,
                mark=none]
         {data/parallel_10_153.svg};
\addlegendentry{Разбиение на 10}
\end{axis}
\end{tikzpicture}
\caption{Замеры времени работы программы в случае увеличения числа потоков для
         изображения размером 153 пикселей}
\label{graph_153}
\end{figure}

\clearpage

\begin{figure}[!h]
\begin{tikzpicture}
\begin{axis}[axis lines=left,
             width=0.9\textwidth,
             height=0.4\textheight,
             xlabel={Число потоков},
             ylabel={Время (с)},
             xmin=0, xmax=12,
             legend pos=north east,
             ymajorgrids=true,
             xmajorgrids=true,
             grid style=dashed]
\addplot table [domain=0:12,
                x=n,
                y=t,
                col sep=comma,
                mark=none]
         {data/parallel_2_200.csv};
\addlegendentry{Разбиение на 2}
\addplot table [domain=0:12,
                x=n,
                y=t,
                col sep=comma,
                mark=none]
         {data/parallel_4_200.csv};
\addlegendentry{Разбиение на 4}
\addplot table [domain=0:12,
                x=n,
                y=t,
                col sep=comma,
                mark=none]
         {data/parallel_6_200.svg};
\addlegendentry{Разбиение на 6}
\addplot table [domain=0:12,
                x=n,
                y=t,
                col sep=comma,
                mark=none]
         {data/parallel_8_200.svg};
\addlegendentry{Разбиение на 8}
\addplot table [domain=0:12,
                x=n,
                y=t,
                col sep=comma,
                mark=none]
         {data/parallel_10_200.svg};
\addlegendentry{Разбиение на 10}
\end{axis}
\end{tikzpicture}
\caption{Замеры времени работы программы в случае увеличения числа потоков для
         изображения размером 200 пикселей}
\label{graph_200}
\end{figure}

\subsection*{Вывод}
В результате проведения эксперимента было выявлено, что время работы программы
в случае применеия распараллеливания действительно стремится к обратной
зависимости от числа потоков.
При этом, было замечено, что с ростом размера изображения, затрачиваемое время
увеличивается согласно закону $O(n^2)$.

\pagebreak

