\section{Экспериментальная часть}
В данном разеделе будут приедены примеры работы программа, а также проведен
сравнительный анализ адгоритмов при различных ситуациях на основе полученных
данных.

\subsection{Демонстрация работы программы}
На рисунках \ref{main_menu} -- \ref{fmeasure} представлены скриншоты,
демонстрирующие работоспособность программы.

\begin{figure}[!h]
    \includegraphics[width=\textwidth]{menu.png}
    \caption{Демонстрация главного меню программы}
    \label{main_menu}
\end{figure}

\begin{figure}[!h]
    \includegraphics[width=\textwidth]{user_sort.png}
    \caption{Демонстрация сортировки массива, введенного пользователем}
    \label{user_sort}
\end{figure}
\pagebreak

\begin{figure}[!h]
    \includegraphics[width=\textwidth]{test.png}
    \caption{Демонстрация тестирования функции сортироки}
    \label{ftest}
\end{figure}
\pagebreak

\begin{figure}[!h]
    \includegraphics[width=\textwidth]{measure.png}
    \caption{Демонстрация проведения замера процессорного времени функции
             сортироки}
    \label{fmeasure}
\end{figure}

\subsection{Технические характеристики}

Технические характеристики устройства, на котором выполнялось тестирование.

\begin{itemize}
    \item Операционная система: Manjaro 5.15 \cite{Manjaro} Linux \cite{Linux} x86\_64.
    \item Память: 16 Гб.
    \item Процессор: AMD Ryzen™ 5 4600H CPU @ 3.0G ГГц \cite{AMD_CPU}.
\end{itemize}

При тестировании ноутбук был включен в сеть электропитания. Во время
тестирования ноутбук был нагружен только встроенными приложениями окружения,
а также системой тестирования.

\subsection{Время выполнения реализаций алгоритмов}

На рисунках \ref{graph_best}, \ref{graph_worst} и \ref{graph_rand} представлены
зависимости времени выполнения реализаций алгоритмов сортировки в лучшем,
худшем и произвольном случаях соответственно.

\begin{figure}[!h]
\begin{tikzpicture}
\begin{axis}[axis lines=left,
             width=0.9\textwidth,
             height=0.4\textheight,
             xlabel={Размер массива},
             ylabel={Время (мс)},
             xmin=10, xmax=1000,
             legend pos=north west,
             ymajorgrids=true,
             xmajorgrids=true,
             grid style=dashed]
\addplot table [domain=10:1000,
                x=n,
                y=tm,
                col sep=comma,
                mark=none]
         {csv/m_gnome_asc.csv};
\addlegendentry{Гномья сортировка}
\addplot table [domain=10:1000,
                x=n,
                y=tm,
                col sep=comma,
                mark=none]
         {csv/m_shaker_asc.csv};
\addlegendentry{Сортировка перемешиванием}
\addplot table [domain=10:1000,
                x=n,
                y=tm,
                col sep=comma,
                mark=none]
         {csv/m_shell_asc.csv};
\addlegendentry{Сортировка Шелла}
\end{axis}
\end{tikzpicture}
\caption{Замеры времени работы реализаций алгоритмов в лучшем случае (упорядоченный массив)}
\label{graph_best}
\end{figure}

\begin{figure}[!h]
\begin{tikzpicture}
\begin{axis}[axis lines=left,
             width=0.9\textwidth,
             height=0.4\textheight,
             xlabel={Размер массива},
             ylabel={Время (мс)},
             xmin=10, xmax=1000,
             legend pos=north west,
             ymajorgrids=true,
             xmajorgrids=true,
             grid style=dashed]
\addplot table [domain=10:1000,
                x=n,
                y=tm,
                col sep=comma,
                mark=none]
         {csv/m_gnome_desc.csv};
\addlegendentry{Гномья сортировка}
\addplot table [domain=10:1000,
                x=n,
                y=tm,
                col sep=comma,
                mark=none]
         {csv/m_shaker_desc.csv};
\addlegendentry{Сортировка перемешиванием}
\addplot table [domain=10:1000,
                x=n,
                y=tm,
                col sep=comma,
                mark=none]
         {csv/m_shell_desc.csv};
\addlegendentry{Сортировка Шелла}
\end{axis}
\end{tikzpicture}
\caption{Замеры времени работы реализаций алгоритмов в худшем случае (обратно упорядоченный массив)}
\label{graph_worst}
\end{figure}
\pagebreak

\begin{figure}[!h]
\begin{tikzpicture}
\begin{axis}[axis lines=left,
             width=0.9\textwidth,
             height=0.4\textheight,
             xlabel={Размер массива},
             ylabel={Время (мс)},
             xmin=10, xmax=1000,
             legend pos=north west,
             ymajorgrids=true,
             xmajorgrids=true,
             grid style=dashed]
\addplot table [domain=10:1000,
                x=n,
                y=tm,
                col sep=comma,
                mark=none]
         {csv/m_gnome_rand.csv};
\addlegendentry{Гномья сортировка}
\addplot table [domain=10:1000,
                x=n,
                y=tm,
                col sep=comma,
                mark=none]
         {csv/m_shaker_rand.csv};
\addlegendentry{Сортировка перемешиванием}
\addplot table [domain=10:1000,
                x=n,
                y=tm,
                col sep=comma,
                mark=none]
         {csv/m_shell_rand.csv};
\addlegendentry{Сортировка Шелла}
\end{axis}
\end{tikzpicture}
\caption{Замеры времени работы реализаций алгоритмов для произвольного массива}
\label{graph_rand}
\end{figure}

\subsection*{Вывод}
По результатам замеров было выявлено, что реализация алгоритма сортировки Шелла
показывает более низкие затраты времени в случаях упорядоченного по убыванию и
произвольного массивов, за ним идет алгоритм сортировки перемешиванием и гномья
соритровка соответственно. Однако в случае упорядоченного набора данных
результаты замеров для сортировки Шелла оказались более высокими, чем у
остальных. В данном случае не было замечено различий во времени работы
реализаций сортировки перемешиванием и гномьей сортировки.

\pagebreak

