\leftsection{Конструкторская часть}
В данном разделе будут рассмотрены требования к программному обеспечению,
некоторые расчетные соотношения и схемы разработанных алгоритмов.

\subsection{Требования к программному обеспечению}
Программа должна предоставлять следующий функционал:
\begin{itemize}
    \item возможность интерактивного изменения положения объектов и источников
          света в сцене;
    \item возможность интерактивного добавления/удаления объектов в сцене;
    \item возможность интерактивного изменения свойств объектов (коэффициент
          восприятия света, коэффициент преломления, тип материала объекта,
          интенсивность и цвет излучения);
    \item возможность интерактивного масштабирования и поворота объектов сцены.
\end{itemize}
\vspace{\baselineskip}

К программе предоставляются следующие требования:
\begin{itemize}
    \item программа должна корректно реагировать на любые действия пользователя;
    \item программа должна предоставлять приемлемое время отклика
          (не более 1 сек) при работе в интерактивном режиме.
\end{itemize}

\subsection{Расчет лучей на поверхности тела}
Выражение для нахождения отраженного $w_{re}$ и преломленного $w_r$ лучей, могут
быть найдены при помощи заданных лучей падения $w_i$ и нормали $n$
(рисунок \ref{fig:edge_ray}).

Для упрощения вычислений работа будет производиться с единичными векторами
$\widehat{w_{re}}$, $\widehat{w_r}$, $\widehat{w_i}$ и $\widehat{n}$.

\subsubsection*{Отраженный луч}
Отраженный луч может быть представлен в виде комбинации векторов падения и
нормали
\[ \widehat{w_{re}} = \alpha \widehat{w_i} + \beta \widehat{n}. \]

Из равенства углов падения и отражения:
\[ \widehat{w_{re}} \widehat{n} = \widehat{w_i} \widehat{n}
   = \alpha \widehat{w_i} \widehat{n} + \beta \widehat{n}^2
   = \alpha \widehat{w_i} \widehat{n} + \beta \]

\[ \beta = \widehat{w_i} \widehat{n} - \alpha \widehat{w_i} \widehat{n}. \]

Так как вектора единичные
\begin{align*}
    |\widehat{w_{re}}| & = |\widehat{w_i}| = 1 \quad \uparrow^2 \\
    |\widehat{w_{re}}|^2 & = |\widehat{w_i}|^2 = 1,
\end{align*}

\begin{align*}
    1 & = {w_{re}}^2 = (\alpha \widehat{w_i} + \beta \widehat{n})^2
        = \alpha^2 \widehat{w_i}^2 + 2 \alpha \beta \widehat{w_i} \widehat{n}
          + \beta^2 \widehat{n}^2
        = \alpha^2 + 2 \alpha \beta \widehat{w_i} \widehat{n} + \beta^2 = \\
      & = \alpha^2 + 2 \alpha (\widehat{w_i} \widehat{n}
                               - \alpha \widehat{w_i} \widehat{n})
                     \widehat{w_i} \widehat{n} + (\widehat{w_i} \widehat{n}
                                                  - \alpha \widehat{w_i} \widehat{n})^2 = \\
      & = \alpha^2 
          + \underline{2 \alpha (\widehat{w_i} \widehat{n})^2}
          - 2 \alpha^2 (\widehat{w_i} \widehat{n})^2
          + (\widehat{w_i} \widehat{n})^2
          - \underline{2 \alpha (\widehat{w_i} \widehat{n})^2}
          + \alpha^2 (\widehat{w_i} \widehat{n})^2 = \\
      & = (\widehat{w_i} \widehat{n})^2
          + (1 - (\widehat{w_i} \widehat{n})^2) \alpha^2,
\end{align*}

\[ 1 - (\widehat{w_i} \widehat{n})^2 = (1 - (\widehat{w_i} \widehat{n})^2) \alpha^2. \]

Если $1 - (\widehat{w_i} \widehat{n})^2 = 0
      \quad \Rightarrow \quad \widehat{w_i} \widehat{n} = \pm 1
      \quad \Rightarrow \quad \widehat{w_i} || \widehat{n}
      \quad \Rightarrow \quad w_{re} = w_i$, \\
иначе $\alpha^2 = 1 \quad \Rightarrow \quad \alpha = \pm 1$, так как падающий
и отраженный луч не могут находиться по одну сторону от нормали, то
$\alpha = -1$, $\beta = 2 \widehat{w_i} \widehat{n}$, а
$\widehat{w_{re}} = 2 (\widehat{w_i} \widehat{n}) \widehat{n} - \widehat{w_i}$.

Таким образом
\begin{equation}
    w_{re} = \widehat{w_{re}} |w_i|
           = (2 (\widehat{w_i} \widehat{n}) \widehat{n} - \widehat{w_i}) |w_i|
           = (2 \frac{w_i n}{|w_i| |n|} \frac{n}{|n|} - \frac{w_i}{|w_i|}) |w_i|
           = 2 \frac{w_i n}{n^2} n - w_i.
\end{equation}

\subsubsection*{Преломленный луч}
Аналогично для преломленного луча
\[ \widehat{w_r} = \alpha \widehat{w_i} + \beta \widehat{n}. \]

По закону Снеллиуса
\begin{align} \label{math:snell}
    n_1 \sin(\theta_i) & = n_2 \sin(\theta_r) \quad \uparrow^2 \nonumber \\
    n_1^2 \sin^2(\theta_i) & = n_2^2 \sin^2(\theta_r) \nonumber \\
    n_1^2 (1 - \cos^2(\theta_i)) & = n_2^2 (1 - \cos^2(\theta_r)).
\end{align}

Так как $\cos{\theta_i} = \widehat{w_i} \widehat{n}$ и
$\cos{\theta_r} = \widehat{w_r} \widehat{n}$, то выражение \refeq{math:snell}
преобразуется в
\begin{align} \label{math:snell2}
    n_1^2 (1 - (\widehat{w_i} \widehat{n})^2) & = n_2^2 (1 - (\widehat{w_r} \widehat{n})^2)
                                                = n_2^2 (1 - (\alpha \widehat{w_i} \widehat{n} + \beta \widehat{n}^2)^2)) = \nonumber \\
                                              & = n_2^2 (1 - (\alpha^2 (\widehat{w_i} \widehat{n})^2
                                                         + 2 \alpha \beta (\widehat{w_i} \widehat{n}) + \beta^2))
\end{align}

Так как векторы $\widehat{w_i}$ и $\widehat{w_r}$ единичные
\begin{equation} \label{math:w_r2}
    1 = \widehat{w_r}^2 = (\alpha \widehat{w_i} + \beta \widehat{n})^2
      = \alpha^2 + 2 \alpha \beta (\widehat{w_i} \widehat{n}) + \beta^2,
\end{equation}
\begin{equation} \label{math:2ab}
    2 \alpha \beta (\widehat{w_i} \widehat{n}) = 1 - \alpha^2 - \beta^2.
\end{equation}

Подставим \refeq{math:2ab} в \refeq{math:snell2}
\begin{align*}
    n_1^2 (1 - (\widehat{w_i} \widehat{n})^2) & = n_2^2 (1 - (\alpha^2 (\widehat{w_i} \widehat{n})^2
                                                              + 1 - \alpha^2 - \beta^2 + \beta^2)) \\
    n_1^2 (1 - (\widehat{w_i} \widehat{n})^2) & = n_2^2 (1 - (\widehat{w_i} \widehat{n})^2) \alpha^2.
\end{align*}

Аналогично, если $1 - (\widehat{w_i} \widehat{n})^2 = 0$, то $w_{re} = -w_i$,
иначе $\alpha = \pm\frac{n_1}{n_2}$. Так как падающий и преломленный лучи не могут
находиться по одну сторону от нормали, то $\alpha = - \frac{n_1}{n_2}$.
Подставим полученное значение в \refeq{math:w_r2}
\begin{equation*}
    \beta^2 - 2 \frac{n_1}{n_2}(\widehat{w_i}\widehat{n})\beta + \frac{n_1^2 - n_2^2}{n_2^2} = 0
\end{equation*}
\begin{equation*}
    \mathrm{D} = 4 \frac{n_1^2}{n_2^2}(\widehat{w_i} \widehat{n})^2 - 4 \frac{n_1^2 - n_2^2}{n_2^2}
               = \frac{4}{n_2^2}(n_2^2 - n_1^2(1 - (\widehat{w_i}\widehat{n})^2)).
\end{equation*}

Пусть $\mathrm{D'} = n_2^2 - n_1^2(1 - (\widehat{w_i}\widehat{n})^2)$, тогда
$\mathrm{D} = \frac{4}{n_2^2}\mathrm{D}$. В силу того, что $\frac{4}{n_2^2} > 0$,
сравнение знака дискриминанта можно проводить с величиной $\mathrm{D'}$.

Если $\mathrm{D'} < 0$ - преломления нет.

Если $\mathrm{D'} = 0$
\begin{align}
    \beta = \frac{n_1}{n_2}(\widehat{w_i}\widehat{n}) \quad &
    \Rightarrow\quad \widehat{w_r} = -\frac{n_1}{n_2}\widehat{w_i} + \frac{n_1}{n_2}(\widehat{w_i}\widehat{n}) \widehat{n} \quad\Rightarrow \nonumber \\
    & \Rightarrow\quad w_r = -\frac{n_1}{n_2} w_i + \frac{n_1}{n_2}\frac{w_i n}{n^2}n = \frac{n_1}{n_2}\Big( \frac{w_i n}{n^2}n - w_i\Big).
\end{align}

Если $\mathrm{D'} > 0$
\begin{equation*}
    \beta = \frac{2 \frac{n_1}{n_2}(\widehat{w_i}\widehat{n}) \pm \sqrt{\mathrm{D}}}{2}
          = \frac{2 \frac{n_1}{n_2}(\widehat{w_i}\widehat{n}) \pm \frac{2}{n_2}\sqrt{\mathrm{D'}}}{2}
          = \frac{n_1(\widehat{w_i}\widehat{n}) \pm \sqrt{\mathrm{D'}}}{n_2}
\end{equation*}

Дляопределения знака подставим $\alpha$ и $\beta$ в выражение для $\widehat{w_r}$ и
проверим, что $\widehat{w_r}$ и $\widehat{n}$ расположены по разные стороны от
границы пересечения, то есть $\widehat{w_r} \widehat{n} < 0$
\begin{align*}
    \widehat{w_r} \widehat{n} = & (-\frac{n_1}{n_2}\widehat{w_i} + \frac{n_1(\widehat{w_i}\widehat{n}) \pm \sqrt{\mathrm{D'}}}{n_2}\widehat{n})\widehat{n} < 0 \\
    & -\frac{n_1}{n_2}(\widehat{w_i}\widehat{n}) + \frac{n_1}{n_2}(\widehat{w_i}\widehat{n}) \pm \frac{\sqrt{\mathrm{D'}}}{n_2} = \pm \frac{\sqrt{\mathrm{D'}}}{n_2} < 0 \quad\Rightarrow \\
    & \Rightarrow\quad \beta = \frac{n_1(\widehat{w_i}\widehat{n}) - \sqrt{\mathrm{D'}}}{n_2} \quad\Rightarrow\quad
      \widehat{w_r} = -\frac{n_1}{n_2}\widehat{w_i} + \frac{n_1(\widehat{w_i} \widehat{n}) - \sqrt{\mathrm{D'}}}{n_2}\widehat{n}.
\end{align*}

Таким образом итоговое значение $w_r$
\begin{equation}
    w_r = \frac{n_1}{n_2}\Big( \frac{w_i n}{n^2}n - w_i\Big) - \frac{|w_i|}{|n|} \frac{\mathrm{D'}}{n_2} n.
\end{equation}

\subsection{Разработка алгоритмов}
На рисунках \ref{fig:algorithm_rt} -- \ref{fig:algorithm_bssrdf} представлены
схемы рассматриваемых алгоритмов.

\clearpage
\vspace*{\fill}
\begin{figure}[h]
    \centering
    \def\svgwidth{0.95\textwidth}
    \input{algorithm_rt.pdf_tex}
    \caption{Схема алгоритма трассировки путей}
    \label{fig:algorithm_rt}
\end{figure}
\vfill
\clearpage

\clearpage
\vspace*{\fill}
\begin{figure}[h]
    \centering
    \def\svgwidth{0.6\textwidth}
    \input{algorithm_bsdf.pdf_tex}
    \caption{Схема алгоритма вычисления функции рассеивания}
    \label{fig:algorithm_bsdf}
\end{figure}
\vfill
\clearpage

\clearpage
\vspace*{\fill}
\begin{figure}[h]
    \centering
    \def\svgwidth{0.2\textwidth}
    \input{algorithm_bssrdf.pdf_tex}
    \caption{Схема алгоритма вычисления функции подповерхностного рассеивания}
    \label{fig:algorithm_bssrdf}
\end{figure}
\vfill
\clearpage

\subsection{Выбор типов и структур данных}

Для реализуемого программного обеспечения необходимо реализовать следующия
структуры данных:
\begin{itemize}
    \item точка --- набор вещественных координат \verb|x, y[, z]|;
    \item вектор --- набор вещественных координат \verb|x, y[, z]|;
    \item нормаль --- набор вещественных координат \verb|x, y[, z]|;
    \item луч --- совокупность точки и вектора;
    \item матрица преобразования --- матрица размера 4x4 для проведения
          преобразований в однородных координатах;
    \item интенсивность --- набор вещественных составляющих, соответствующих
          красному, зеденому и синему цветам соответственно;
    \item цвет --- нормализованная интенсивность;
    \item объект --- набор математических обыектов, соответствующих
          разделу~\ref{formalisation} и две матрицы преобразования,
          соответствующие локальным преобразованиям и преобразованию
          базиса;
    \item текстура --- матрица инетнсивностей.
\end{itemize}
Для обеспечения возможности построения сложных объектов из набора примитивов,
объекты также должны содержать ссылки на своего родителя и предков.

\subsection*{Вывод}
В данном разделе были рассмотрены требования к программному обеспечению,
некоторые расчетные соотношения, схемы разработанных алгоритмов и необходимые
структуры данных.

\pagebreak

