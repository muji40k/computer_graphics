\section*{Введение}
\addcontentsline{toc}{section}{Введение}

Компьютерная графика является неотъемлемой частью современного мира в
особенности она востребована в сфере кинопроизводства и игровой
промышленности. Именно поэтому синтез реалистического изображения является
одной из важнейших задач.

Подобные условия требуют создания средств и методик, учитывающих такие
оптические явления как преломление, отражение и рассеивание света, зависящие от
свойств визуализируемых объектов. Но рост точности и детальности
разрабатываемых алгоритмов зачастую приводит к более высоким затратам по
времени и памяти.

Целью данной работы является разработка программы, позволяющей создать
реалистическое изображение с учетом подповерхностного рассеивания, на основе
трехмерной сцены наполненной объектами (сфера, куб, конус, цилиндр,
полигональная модель) и источниками света, положение, количество и свойства
которых задается пользователем.

Задачи:
\begin{enumerate}
    \item выявить неотъемлемые качества реалистического изображения;
    \item провести анализ существующих алгоритмов;
    \item выбрать оптимальные пути для решения основной задачи;
    \item реализовать выбранные алгоритмы;
    \item создать программный продукт для решения задачи;
    \item исследовать время работы полученной программы в случае
          распараллеливания алгоритма.
\end{enumerate}

\pagebreak

