\section*{Введение}
\addcontentsline{toc}{section}{Введение}

Компьютерная графика является неотъемлемой частью современного мира в
особенности она востребована в сфере кинопроизводства и игровой
промышленности. Именно поэтому синтез реалистического изображения является
одной из важнейших задач.

Подобные условия требуют создания средств и методик, учитывающих такие
оптические явления как преломление, отражение и рассеивание света, зависящие от
свойств визуализируемых объектов. Но рост точности и детальности
разрабатываемых алгоритмов зачастую приводит к более высоким затратам по
времени и памяти.

\emph{Целью} данной работы является разработка программы, позволяющей создать
реалистическое изображение с учетом подповерхностного рассеивания, на основе
трехмерной сцены наполненной объектами (сфера, куб, конус, цилиндр,
полигональная модель) и источниками света, положение, количество и свойства
которых задается пользователем.

{\large\emph{Задачи.}}
\begin{enumerate}
    \item Выявить неотъемлемые качества реалистического изображения.
    \item Провести анализ существующих алгоритмов.
    \item Выбрать оптимальные пути для решения основной задачи.
    \item Реализовать выбранные алгоритмы.
    \item Создать программный продукт для решения задачи.
\end{enumerate}

\pagebreak

