\section*{Заключение}
\addcontentsline{toc}{section}{Заключение}

В ходе выполнения данной работы была разработана программа, позволяющей создать
реалистическое изображение без учета подповерхностного рассеивания, на основе
трехмерной сцены наполненной объектами (сфера, куб, конус, цилиндр,
полигональная модель) и источниками света, положение, количество и свойства
которых может задаваться пользователем интерактивно.

Были решены следующие задачи:
\begin{itemize}
    \item выявлены неотъемлемые качества реалистического изображения;
    \item проведен анализ существующих алгоритмов;
    \item выбран путь для решения основной задачи;
    \item реализованы выбранные алгоритмы;
    \item создан программный продукт для решения задачи.
    \item исследовано время работы полученной программы в случае
          распараллеливания алгоритма.
\end{itemize}

