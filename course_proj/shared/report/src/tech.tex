\section{Технологическая часть}
В данном разделе будут рассмотрены реализации алгоритмов сортировки.

\subsection{Средства реализации}
Для реализации алгоритмов в данной работе был выбран язык C, так как
данный язык предоставляет возможность решить поставленную задачу и произвести
замеры процессорного времени. В качестве компилятора для программы используется
GNU GCC.

Так как работа требует измерения процессорного времени, была подобрана
соответствующая функция из библиотеки \verb+time+ - \verb+clock()+
\cite{CPUtime}.

\subsection{Сведения о модулях программы}
Программа состоит из трех модулей:
\begin{itemize}
    \item \verb+main.c+ - главный модуль программы, содержит реализации
          подпрограмм взаимодействия с пользователем и замера процессорного
          времени для алгоритмов сортировки;
    \item \verb+sorts.c+ - содержит реализации алгоритмов сортировки;
    \item \verb+cli.c+ - содержит реализиции подпрограмм создания интерфейса
          коммандной строки.
\end{itemize}

\subsection{Реализации алгоритмов}
На листингах \ref{code_gnome}, \ref{code_shaker} и \ref{code_shell} представлены
реализации алгоритмов сортировки.

\lstinputlisting[language=c, label={code_gnome}, caption={Листинг кода Гномьей сортировки}]{gnome.c}

\lstinputlisting[language=c, label={code_shaker}, caption={Листинг кода Сортировки перемешиванием}]{shaker.c}
\vfill
\pagebreak

\vspace*{\fill}
\lstinputlisting[language=c, label={code_shell}, caption={Листинг кода Сортировки Шелла}]{shell.c}
\vfill
\pagebreak

\subsection{Способ замера процессорного времени}
Процессорное время, затраченное реализацией алгоритма для массива определенной
длины, рассчитывается как среднее арифметическое из нескольких замеров: 
замеров времени для одного массива и множества массивов с различными значениями.
Реализация функции, решающей подобную задачу, представлена на листинге
\ref{code_measure}. Для получения временного промежутка, указанная выше функция
\verb+clock()+, должна вызываться два раза - в начале и в конце исследуемого
интервала.

\lstinputlisting[language=c, label={code_measure}, caption={Листинг кода функции замера процессорного времени}]{measure.c}
\pagebreak

\subsection{Тестирование}
В таблице \ref{test_table} представлены тестовые данные для алгоритмов
сортировки. Все тесты пройдены успешно.

\begin{center}
    \begin{threeparttable}[h]
        \caption{Данные для тестирования алгоритмов сортировки}
        \label{test_table}
        \centering
        \footnotesize
        \begin{tabular}{| c | c | c | c |}
            \hline
            Описание            & № & Входные данные                            & Ожидаемый результат                       \\ \hline
            Массив упорядочен   & 1 & [2, 2, 2, 7, 17, 23, 25, 34, 40, 41]      & [2, 2, 2, 7, 17, 23, 25, 34, 40, 41]      \\
                                & 2 & [-11, -6, -2, 0, 5, 8, 11, 15, 18, 22]    & [-11, -6, -2, 0, 5, 8, 11, 15, 18, 22]    \\ \hline
            Массив обратно      & 3 & [51, 45, 43, 37, 35, 25, 21, 15, 8, 6]    & [6, 8, 15, 21, 25, 35, 37, 43, 45, 51]    \\
            упорядочен          & 4 & [7, 4, 2, 2, -6, -12, -19, -20, -24, -31] & [-31, -24, -20, -19, -12, -6, 2, 2, 4, 7] \\ \hline
            Произвольный массив & 5 & [84, 4, -75, 63, 67, 10, 15, -48, 68, 69] & [-75, -48, 4, 10, 15, 63, 67, 68, 69, 84] \\
                                & 6 & [-6, 74, 1, 57, 19, -19, 74, 5, 89, -45]  & [-45, -19, -6, 1, 5, 19, 57, 74, 74, 89]  \\ \hline
        \end{tabular}
    \end{threeparttable}
\end{center}

\subsection*{Вывод}
В данном разделе были рассмотрены реализации алгоритмов сортировки.

\pagebreak

